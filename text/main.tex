\documentclass[a4paper,12pt]{article}

\usepackage[style=chicago-authordate]{biblatex}
\addbibresource{../bibfile.bib}

\usepackage[brazil]{babel}
\usepackage{amsmath}
\usepackage{geometry}
\usepackage{multirow, booktabs}
\usepackage{graphicx}
\usepackage{setspace}
\usepackage{float}
\usepackage{fancyhdr}
\usepackage{gensymb}
\usepackage{hyperref}
\usepackage{listings}
\usepackage{xcolor}
\usepackage{microtype}

\lstset{
    language=Octave,
    frame=single,
    numbers=left,
    numberstyle=\tiny,
    stepnumber=1,
    numbersep=5pt,
    backgroundcolor=\color{gray!10},
    showspaces=false,
    showstringspaces=false,
    showtabs=false,
    tabsize=4,
    captionpos=b,
    breaklines=true,
    breakatwhitespace=false,
    title=\lstname,
    basicstyle=\ttfamily,
    keywordstyle=\color{blue},
    commentstyle=\color{green!50!black},
    stringstyle=\color{red},
}

\pagestyle{fancy}
\fancyhf{}
\lhead{\footnotesize Cefet-MG: L06CSD - Solução da lista de exercícios VII}
\cfoot{\footnotesize \thepage}

\setlength{\parindent}{0in}

\renewcommand{\arraystretch}{1.5} % Ajusta o espaçamento entre as linhas do array

\begin{document}

%%%%%%%%%%%%%%%%%%%%%%%%%%%%%%%%%%%%%%%%%%%%%%%%
% Seção de Título
%%%%%%%%%%%%%%%%%%%%%%%%%%%%%%%%%%%%%%%%%%%%%%%%

    \thispagestyle{empty} % Desativa cabeçalho na primeira página

    \begin{tabular}{p{15.5cm}}
    {\large \textbf{Controle de Sistemas Dinâmicos} } \\
    Centro Federal de Educação Tecnológica de Minas Gerais \\
    02 de dezembro de 2024 \\ Campus Timóteo \\
    \hline
    \\
    \end{tabular}

    \vspace*{0.3cm}

    \begin{center}
    {\Large \textbf{Resolução da lista de exercícios VII}}
        \vspace{2mm}

        {\textbf{Eliel Vitor Almeida \\ João Pedro Ferreira Duarte \\ Marcos Vinícius de Oliveira Silva }}
    \end{center}

    \vspace{0.4cm}

%%%%%%%%%%%%%%%%%%%%%%%%%%%%%%%%%%%%%%%%%%%%%%%%
% Corpo do Documento
%%%%%%%%%%%%%%%%%%%%%%%%%%%%%%%%%%%%%%%%%%%%%%%%

    Em sequência, estão os comandos e resoluções das questões da avaliação.

    \begin{enumerate}
        \item Crie as seguintes funções de transferência na forma polinomial:
        \[
            \begin{array}{ccc}
                \frac{3}{2s + 1} & \frac{3}{2s - 1} & \frac{5}{(s + 2) \cdot (s + 5)} \\
                \frac{5}{(s - 2) \cdot (s - 5)} & \frac{5}{s^2 + 2s + 5} & \frac{5}{s^2 - 2s + 5} \\
                \frac{5}{s^2 + 16} & \frac{5}{s^2 - 16} & \frac{5}{s^2 + 6s + 9} \\
            \end{array}
        \]

        \begin{enumerate}
            \item Calcule os pólos de cada função de transferência.
            \item Plote a resposta ao degrau de cada função de transferência. Use a função \texttt{step}.
            \item Descreva o comportamento de cada resposta obtida, fale sobre a estabilidade do sistema e relacione este comportamento aos pólos. Comente e conclua.
        \end{enumerate}
        Em resposta aos itens anteriores:
        \vspace{0.5cm}
        \begin{lstlisting}
pkg load control;

numerador_A = 3;
denominador_A = [2, 1];

numerador_B = 3;
denominador_B = [2, -1];

numerador_C = 5;
denominador_C = [-2,-5];

numerador_D = 5;
denominador_D = [2,5];

numerador_E = 5;
denominador_E = [1,2,5];

numerador_F = 5;
denominador_F = [1,-2,5];

numerador_G = 5;
denominador_G = [1,0,16];

numerador_H = 5;
denominador_H = [1,0,-16];

numerador_I = 5;
denominador_I = [1,6,9];
% Calcular os pólos (raízes do denominador)
poles_A = roots(denominador_A);
poles_B = roots(denominador_B);
poles_C = roots(denominador_C);
poles_D = roots(denominador_D);
poles_E = roots(denominador_E);
poles_F = roots(denominador_F);
poles_G = roots(denominador_G);
poles_H = roots(denominador_H);
poles_I = roots(denominador_I);

% Exibir os pólos
disp('Pólos de A:');
disp(poles_A);

disp('Pólos de B:');
disp(poles_B);

disp('Pólos de C:');
disp(poles_C);

disp('Pólos de D:');
disp(poles_D);

disp('Pólos de E:');
disp(poles_E);

disp('Pólos de F:');
disp(poles_F);

disp('Pólos de G:');
disp(poles_G);

disp('Pólos de H:');
disp(poles_H);

disp('Pólos de I:');
disp(poles_I);

%Plotar no grafico

plot_A = tf(numerador_A,denominador_A);
step(plot_A,10);

plot_B = tf(numerador_B,denominador_B);
step(plot_B,10);

plot_C = tf(numerador_C,denominador_C);
step(plot_C,10);

plot_D = tf(numerador_D,denominador_D);
step(plot_D,10);

plot_E = tf(numerador_E,denominador_E);
step(plot_E,10);

plot_F = tf(numerador_F,denominador_F);
step(plot_F,10);

plot_G = tf(numerador_G,denominador_G);
step(plot_G,10);

plot_H = tf(numerador_H,denominador_H);
step(plot_H,10);

plot_I = tf(numerador_I,denominador_I);
step(plot_I,10);
        \end{lstlisting}


        \item Sejam os sistemas representados pelas seguintes funções:
        \[
            \begin{array}{cc}
                G_1 = \frac{1}{2s + 1}, & G_2 = \frac{1}{s^2 + 0.5s + 1}
            \end{array}
        \]
        \begin{enumerate}
            \item Verifique como tais sistemas respondem às seguintes entradas:
            \begin{enumerate}
                \item Rampa.
                \item Impulso.
                \item Plote o gráfico de cada uma das funções.
            \end{enumerate}
        \end{enumerate}


        \item Considere o sistema com realimentação descrito na figura abaixo:
        \begin{enumerate}
            \item Calcule a função de transferência em malha fechada usando as funções \texttt{series} e \texttt{feedback}.
            \item Obtenha a resposta ao degrau unitário do sistema em malha fechada com a função \texttt{step} e verifique que o valor final da saída é $\frac{2}{5}$.
        \end{enumerate}

        \item Um sistema possui a seguinte função de transferência:
        \begin{equation}
            \frac{X(s)}{R(s)} = \frac{\frac{20}{z} \cdot (s + z)}{s^2 + 3s + 20}
        \end{equation}
        \begin{enumerate}
            \item Obtenha a resposta ao degrau unitário do sistema para o parâmetro \(z = 5\), \(z = 10\), e \(z = 15\).
            \item Plote as 3 curvas no mesmo gráfico. Compare, comente e conclua.
        \end{enumerate}
    \end{enumerate}

%%%%%%%%%%%%%%%%%%%%%%%%%%%%%%%%%%%%%%%%%%%%%%%%
% Referências e Bibliografia
%%%%%%%%%%%%%%%%%%%%%%%%%%%%%%%%%%%%%%%%%%%%%%%%

    \clearpage
    \printbibliography

\end{document}
